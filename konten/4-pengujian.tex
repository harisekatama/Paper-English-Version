% Ubah judul dan label berikut sesuai dengan yang diinginkan.
\section{Hasil Pengujian}
\label{sec:hasil pengujian}

% Ubah paragraf-paragraf pada bagian ini sesuai dengan yang diinginkan.
Pada bab ini akan dipaparkan mengenai beberapa skenario pengujian sesuai dengan telah dijelaskan pada metodologi. Skenario pengujian ini dilakukan guna untuk mengetahui waktu \emph{delay} yang dibutuhkan untuk mentransmisikan data dari laptop menuju ESP32. Skenario yang nantinya akan diterapkan pada pengujian meliputi beberapa poin sebagai berikut:

\begin{enumerate}
  \item Pengujian waktu \emph{delay} pengiriman data String melalui Bluetooth
  \item Pengujian waktu \emph{delay} pengiriman data JSON melalui Bluetooth
  \item Pengujian waktu \emph{delay} pengiriman data String melalui \emph{Access Point} WiFi
  \item Pengujian waktu \emph{delay} pengiriman data JSON melalui \emph{Access Point} WiFi
\end{enumerate}

Pelaksanaan metodologi serta skenario pengujian yang akan dipaparkan dalam bab ini diharapkan dapat memberikan pemahaman mengenai hasil dan pembahasan sehingga dapat ditarik kesimpulan dari Tugas Akhir yang telah dilaksanakan.

\subsection{Pengujian Waktu Delay Pengiriman Data String Melalui Bluetooth}

Pengujian waktu \emph{delay} pengiriman data ini dilakukan dengan cara mengirimkan data berupa string dari laptop menuju ESP32 melalui Bluetooth. Data yang dikirimkan adalah data arah dan kecepatan yang dipisahkan dengan koma seperti yang dapat dilihat pada Persamaan \ref{eq:string-2data}.

\begin{equation}
  \label{eq:string-2data}
    Arah(char),Kecepatan(integer)
\end{equation}

Variabel arah memiliki tipe data \emph{char} yang digunakan untuk menentukan arah gerak dari motor kursi roda serta variabel kecepatan memiliki tipe data integer yang akan menentukan kecepatan maksimal dari rotasi motor kursi roda. Hasil dari pengujian waktu \emph{delay} pengiriman data string melalui bluetooth dapat dilihat pada Tabel \ref{tbl:delayBluetooth} dan Tabel \ref{tbl:delayBluetooth1}.

\begin{table}[]
\centering
  \caption{Pengujian Waktu Delay Pengiriman Data String Berisi 2 Nilai Melalui Bluetooth}
  \label{tbl:delayBluetooth}
  \begin{tabular}{|ccc|c|}
  \hline
  \multicolumn{1}{|c|}{Data}  & \multicolumn{1}{c|}{Timestamp Send}  & Timestamp Received & Delay Time  \\ \hline
  \multicolumn{1}{|c|}{C,40}  & \multicolumn{1}{c|}{22:02:06.857715} & 22:02:07.892       & 1,034285    \\ \hline
  \multicolumn{1}{|c|}{A,226} & \multicolumn{1}{c|}{22:02:08.358496} & 22:02:09.390       & 1,031504    \\ \hline
  \multicolumn{1}{|c|}{E,21}  & \multicolumn{1}{c|}{22:02:09.859346} & 22:02:10.921       & 1,061654    \\ \hline
  \multicolumn{1}{|c|}{C,168} & \multicolumn{1}{c|}{22:02:11.360115} & 22:02:12.405       & 1,044885    \\ \hline
  \multicolumn{1}{|c|}{B,247} & \multicolumn{1}{c|}{22:02:12.860932} & 22:02:13.900       & 1,039068    \\ \hline
  \multicolumn{1}{|c|}{B,72}  & \multicolumn{1}{c|}{22:02:14.361692} & 22:02:15.400       & 1,038308    \\ \hline
  \multicolumn{1}{|c|}{E,129} & \multicolumn{1}{c|}{22:05:28.232092} & 22:05:29.283       & 1,050908    \\ \hline
  \multicolumn{1}{|c|}{D,117} & \multicolumn{1}{c|}{22:05:29.732554} & 22:05:30.765       & 1,032446    \\ \hline
  \multicolumn{1}{|c|}{C,31}  & \multicolumn{1}{c|}{22:05:31.233293} & 22:05:32.264       & 1,030707    \\ \hline
  \multicolumn{1}{|c|}{B,248} & \multicolumn{1}{c|}{22:05:32.734151} & 22:05:33.779       & 1,044849    \\ \hline
  \multicolumn{1}{|c|}{D,126} & \multicolumn{1}{c|}{22:05:34.235059} & 22:05:35.271       & 1,035941    \\ \hline
  \multicolumn{1}{|c|}{C,2}   & \multicolumn{1}{c|}{22:07:30.543505} & 22:07:31.599       & 1,055495    \\ \hline
  \multicolumn{1}{|c|}{B,29}  & \multicolumn{1}{c|}{22:07:32.044303} & 22:07:33.077       & 1,032697    \\ \hline
  \multicolumn{1}{|c|}{C,247} & \multicolumn{1}{c|}{22:07:33.545370} & 22:07:34.595       & 1,04963     \\ \hline
  \multicolumn{1}{|c|}{A,190} & \multicolumn{1}{c|}{22:07:35.046573} & 22:07:36.085       & 1,038427    \\ \hline
  \multicolumn{1}{|c|}{A,49}  & \multicolumn{1}{c|}{22:07:36.547315} & 22:07:37.575       & 1,027685    \\ \hline
  \multicolumn{1}{|c|}{B,25}  & \multicolumn{1}{c|}{22:09:44.370413} & 22:09:45.405       & 1,034587    \\ \hline
  \multicolumn{1}{|c|}{D,72}  & \multicolumn{1}{c|}{22:09:45.871428} & 22:09:46.921       & 1,049572    \\ \hline
  \multicolumn{1}{|c|}{C,63}  & \multicolumn{1}{c|}{22:09:47.372407} & 22:09:48.415       & 1,042593    \\ \hline
  \multicolumn{1}{|c|}{A,245} & \multicolumn{1}{c|}{22:09:48.873310} & 22:09:49.885       & 1,01169     \\ \hline
  \multicolumn{1}{|c|}{C,70}  & \multicolumn{1}{c|}{22:09:50.374135} & 22:09:51.406       & 1,031865    \\ \hline
  \multicolumn{1}{|c|}{D,240} & \multicolumn{1}{c|}{22:09:51.874944} & 22:09:52.929       & 1,054056    \\ \hline
  \multicolumn{1}{|c|}{A,46}  & \multicolumn{1}{c|}{22:11:46.717729} & 22:11:47.744       & 1,026271    \\ \hline
  \multicolumn{1}{|c|}{B,245} & \multicolumn{1}{c|}{22:11:48.218593} & 22:11:49.263       & 1,044407    \\ \hline
  \multicolumn{1}{|c|}{B,68}  & \multicolumn{1}{c|}{22:11:49.719513} & 22:11:50.763       & 1,043487    \\ \hline
  \multicolumn{1}{|c|}{E,204} & \multicolumn{1}{c|}{22:11:51.219962} & 22:11:52.271       & 1,051038    \\ \hline
  \multicolumn{1}{|c|}{B,26}  & \multicolumn{1}{c|}{22:11:52.720738} & 22:11:53.746       & 1,025262    \\ \hline
  \multicolumn{1}{|c|}{D,113} & \multicolumn{1}{c|}{22:13:39.843434} & 22:13:40.871       & 1,027566    \\ \hline
  \multicolumn{1}{|c|}{B,195} & \multicolumn{1}{c|}{22:13:41.344420} & 22:13:42.389       & 1,04458     \\ \hline
  \multicolumn{1}{|c|}{A,242} & \multicolumn{1}{c|}{22:13:42.845419} & 22:13:43.876       & 1,030581    \\ \hline
  \multicolumn{1}{|c|}{A,17}  & \multicolumn{1}{c|}{22:13:44.346312} & 22:13:45.360       & 1,013688    \\ \hline
  \multicolumn{1}{|c|}{A,140} & \multicolumn{1}{c|}{22:13:45.847280} & 22:13:46.915       & 1,06772     \\ \hline
  \multicolumn{3}{|c|}{Average Delay Time}                                                & 1,038982875 \\ \hline
  \end{tabular}
\end{table}

\newpage

Tabel \ref{tbl:delayBluetooth} menampilkan pengiriman data string yang terdiri dari 2 nilai dan dipisahkan dengan simbol koma (","). Dalam pengujian kali ini, data dikirimkan sebanyak 32 kali secara berturut-turut dengan penambahan waktu \emph{delay} sebesar 1.5 detik setiap kali mengirimkan data. Hal ini dilakukan agar ESP32 dapat menerima data dengan baik dan berhasil memisahkan dan memasukkan kedua nilai tersebut sesuai dengan variabel yang telah ditentukan. Hasil dari pengujian ini menunjukkan bahwa waktu pengiriman rata-rata dari laptop menuju ESP32 melalui Bluetooth adalah sebesar 1.038982875 detik.

Tabel \ref{tbl:delayBluetooth1} menampilkan pengiriman data string yang terdiri dari 1 nilai, yaitu arah. Dalam pengujian kali ini, data dikirimkan sebanyak 35 kali secara berturut-turut tanpa penambahan waktu \emph{delay} setiap kali mengirimkan data. Hasil dari pengujian ini menunjukkan bahwa waktu pengiriman rata-rata dari laptop menuju ESP32 melalui Bluetooth adalah sebesar 0.3495228571 detik.

\begin{table}[]
\centering
  \caption{Pengujian Waktu Delay Pengiriman Data String Berisi 1 Nilai Melalui Bluetooth}
  \label{tbl:delayBluetooth1}
  \begin{tabular}{|ccc|c|}
  \hline
  \multicolumn{1}{|c|}{Data} & \multicolumn{1}{c|}{Timestamp Sent}  & Timestamp Received & Delay Time   \\ \hline
  \multicolumn{1}{|c|}{D}    & \multicolumn{1}{c|}{17:56:13.163165} & 17:56:14.352       & 1,189        \\ \hline
  \multicolumn{1}{|c|}{E}    & \multicolumn{1}{c|}{17:56:14.260985} & 17:56:14.491       & 0,230        \\ \hline
  \multicolumn{1}{|c|}{D}    & \multicolumn{1}{c|}{17:56:14.442510} & 17:56:14.708       & 0,265        \\ \hline
  \multicolumn{1}{|c|}{E}    & \multicolumn{1}{c|}{17:56:14.649861} & 17:56:14.878       & 0,228        \\ \hline
  \multicolumn{1}{|c|}{D}    & \multicolumn{1}{c|}{17:56:14.782992} & 17:56:15.017       & 0,234        \\ \hline
  \multicolumn{1}{|c|}{C}    & \multicolumn{1}{c|}{17:56:14.957997} & 17:56:15.188       & 0,230        \\ \hline
  \multicolumn{1}{|c|}{A}    & \multicolumn{1}{c|}{17:56:15.096377} & 17:56:15.311       & 0,215        \\ \hline
  \multicolumn{1}{|c|}{D}    & \multicolumn{1}{c|}{17:56:15.255594} & 17:56:15.483       & 0,227        \\ \hline
  \multicolumn{1}{|c|}{D}    & \multicolumn{1}{c|}{17:56:15.406847} & 17:56:15.745       & 0,338        \\ \hline
  \multicolumn{1}{|c|}{E}    & \multicolumn{1}{c|}{17:56:15.663510} & 17:56:15.930       & 0,266        \\ \hline
  \multicolumn{1}{|c|}{D}    & \multicolumn{1}{c|}{17:56:15.893686} & 17:56:16.195       & 0,301        \\ \hline
  \multicolumn{1}{|c|}{C}    & \multicolumn{1}{c|}{17:56:16.118896} & 17:56:16.320       & 0,201        \\ \hline
  \multicolumn{1}{|c|}{A}    & \multicolumn{1}{c|}{17:56:16.287623} & 17:56:16.458       & 0,170        \\ \hline
  \multicolumn{1}{|c|}{A}    & \multicolumn{1}{c|}{17:56:16.416326} & 17:56:16.674       & 0,258        \\ \hline
  \multicolumn{1}{|c|}{A}    & \multicolumn{1}{c|}{17:56:16.600528} & 17:56:16.862       & 0,261        \\ \hline
  \multicolumn{1}{|c|}{C}    & \multicolumn{1}{c|}{17:56:16.800782} & 17:56:17.080       & 0,279        \\ \hline
  \multicolumn{1}{|c|}{A}    & \multicolumn{1}{c|}{17:56:16.972893} & 17:56:17.360       & 0,387        \\ \hline
  \multicolumn{1}{|c|}{C}    & \multicolumn{1}{c|}{17:56:17.250136} & 17:56:17.641       & 0,391        \\ \hline
  \multicolumn{1}{|c|}{B}    & \multicolumn{1}{c|}{17:56:17.544344} & 17:56:17.779       & 0,235        \\ \hline
  \multicolumn{1}{|c|}{C}    & \multicolumn{1}{c|}{17:56:17.728154} & 17:56:18.012       & 0,284        \\ \hline
  \multicolumn{1}{|c|}{A}    & \multicolumn{1}{c|}{17:56:17.945683} & 17:56:18.151       & 0,205        \\ \hline
  \multicolumn{1}{|c|}{D}    & \multicolumn{1}{c|}{17:56:18.097472} & 17:56:18.382       & 0,285        \\ \hline
  \multicolumn{1}{|c|}{A}    & \multicolumn{1}{c|}{17:56:18.302853} & 17:56:18.614       & 0,311        \\ \hline
  \multicolumn{1}{|c|}{D}    & \multicolumn{1}{c|}{17:56:18.550624} & 17:56:18.818       & 0,267        \\ \hline
  \multicolumn{1}{|c|}{B}    & \multicolumn{1}{c|}{17:56:18.745921} & 17:56:18.990       & 0,244        \\ \hline
  \multicolumn{1}{|c|}{A}    & \multicolumn{1}{c|}{17:56:18.932188} & 17:56:19.162       & 0,230        \\ \hline
  \multicolumn{1}{|c|}{B}    & \multicolumn{1}{c|}{17:56:19.129604} & 17:56:19.596       & 0,466        \\ \hline
  \multicolumn{1}{|c|}{C}    & \multicolumn{1}{c|}{17:56:19.482713} & 17:56:19.937       & 0,454        \\ \hline
  \multicolumn{1}{|c|}{C}    & \multicolumn{1}{c|}{17:56:19.876232} & 17:56:20.341       & 0,465        \\ \hline
  \multicolumn{1}{|c|}{C}    & \multicolumn{1}{c|}{17:56:20.216948} & 17:56:20.649       & 0,432        \\ \hline
  \multicolumn{1}{|c|}{E}    & \multicolumn{1}{c|}{17:56:20.588913} & 17:56:21.052       & 0,463        \\ \hline
  \multicolumn{1}{|c|}{B}    & \multicolumn{1}{c|}{17:56:20.913091} & 17:56:21.468       & 0,555        \\ \hline
  \multicolumn{1}{|c|}{C}    & \multicolumn{1}{c|}{17:56:21.323255} & 17:56:21.873       & 0,550        \\ \hline
  \multicolumn{1}{|c|}{C}    & \multicolumn{1}{c|}{17:56:21.820789} & 17:56:22.278       & 0,457        \\ \hline
  \multicolumn{1}{|c|}{B}    & \multicolumn{1}{c|}{17:56:22.149208} & 17:56:22.807       & 0,658        \\ \hline
  \multicolumn{3}{|c|}{Average Delay Time}                                               & 0,3495228571 \\ \hline
  \end{tabular}
\end{table}

\newpage

Ditemukan perbedaan waktu \emph{delay} yang cukup signifikan antara proses pengiriman string yang mengandung 2 nilai dibandingkan dengan string yang hanya berisi 1 nilai. Saat mengirimkan string yang memuat 2 data, pengujian menunjukkan adanya waktu \emph{delay} sekitar 1.038982875 detik. Dalam perbandingan dengan pengujian yang melibatkan string 1 nilai, waktu \emph{delay} yang tercatat hanya sekitar 0.3495228571 detik. Penting untuk diperhatikan bahwa terdapat kekurangan saat mengirimkan data string yang mengandung 2 nilai, yaitu adanya penambahan \emph{delay} sebesar 1.5 detik setiap kali pengiriman data dilakukan. Hal ini dilakukan agar ESP32 dapat menerima data secara optimal. Oleh karena itu, dapat disimpulkan bahwa dalam konteks pengiriman data melalui Bluetooth, lebih disarankan untuk mengirimkan string dengan hanya 1 nilai guna menghindari \emph{delay} tambahan yang dapat mempengaruhi efisiensi dan kecepatan transmisi data secara keseluruhan.

\subsection{Pengujian Waktu Delay Pengiriman Data JSON Melalui Bluetooth}

Pengujian waktu \emph{delay} pengiriman data ini dilakukan dengan cara mengirimkan data berupa JSON dari laptop menuju ESP32 melalui Bluetooth. Data yang dikirimkan adalah data arah dan kecepatan yang dirangkum menjadi JSON seperti yang dapat dilihat pada Persamaan \ref{eq:json-2data} dan juga JSON yang hanya berisikan data arah seperti pada Persamaan \ref{eq:json-1data}

\begin{equation}
  \label{eq:json-2data}
    \{'arah': arah(char), 'kecepatan': kecepatan(integer)\}
\end{equation}

\begin{equation}
  \label{eq:json-1data}
    \{'arah': arah(char)\}
\end{equation}

Data JSON terdiri dari 2 bagian, yaitu \emph{key} dan \emph{value}. Terdapat 2 \emph{key}, yaitu arah dan kecepatan. \emph{Key} arah berisi \emph{value} dengan tipe data char yang digunakan untuk menentukan arah gerak dari motor kursi roda dan \emph{key} kecepatan berisi \emph{value} dengan tipe data integer yang digunakan untuk menentukan kecepatan maksimal dari rotasi motor kursi roda. Hasil dari pengujian waktu \emph{delay} pengiriman data JSON dapat dilihat pada Tabel \ref{tbl:delayBluetoothJSON2} dan Tabel \ref{tbl:delayBluetoothJSON1}.

\begin{table}[!h]
\centering
  \caption{Pengujian Waktu Delay Pengiriman Data JSON Berisi 2 Nilai Melalui Bluetooth}
  \label{tbl:delayBluetoothJSON2}
  \begin{tabular}{|c|c|}
  \hline
  Data                              & Delay Time  \\ \hline
  \{'arah': 'C', 'kecepatan': 73\}  & 1,070983    \\ \hline
  \{'arah': 'D', 'kecepatan': 69\}  & 1,043823    \\ \hline
  \{'arah': 'D', 'kecepatan': 140\} & 1,077356    \\ \hline
  \{'arah': 'A', 'kecepatan': 112\} & 1,044772    \\ \hline
  \{'arah': 'B', 'kecepatan': 125\} & 1,032068    \\ \hline
  \{'arah': 'A', 'kecepatan': 141\} & 1,068203    \\ \hline
  \{'arah': 'D', 'kecepatan': 210\} & 1,096744    \\ \hline
  \{'arah': 'E', 'kecepatan': 48\}  & 1,050115    \\ \hline
  \{'arah': 'D', 'kecepatan': 247\} & 1,035611    \\ \hline
  \{'arah': 'E', 'kecepatan': 48\}  & 1,081798    \\ \hline
  \{'arah': 'A', 'kecepatan': 117\} & 1,104833    \\ \hline
  \{'arah': 'B', 'kecepatan': 175\} & 1,042016    \\ \hline
  \{'arah': 'C', 'kecepatan': 121\} & 1,026936    \\ \hline
  \{'arah': 'E', 'kecepatan': 55\}  & 1,070894    \\ \hline
  \{'arah': 'A', 'kecepatan': 134\} & 1,097328    \\ \hline
  \{'arah': 'A', 'kecepatan': 241\} & 1,038598    \\ \hline
  \{'arah': 'C', 'kecepatan': 8\}   & 1,041939    \\ \hline
  \{'arah': 'C', 'kecepatan': 242\} & 1,061519    \\ \hline
  \{'arah': 'E', 'kecepatan': 251\} & 1,098049    \\ \hline
  \{'arah': 'C', 'kecepatan': 65\}  & 1,010486    \\ \hline
  \{'arah': 'B', 'kecepatan': 184\} & 1,044790    \\ \hline
  \{'arah': 'D', 'kecepatan': 244\} & 1,065066    \\ \hline
  \{'arah': 'D', 'kecepatan': 216\} & 1,092696    \\ \hline
  \{'arah': 'E', 'kecepatan': 209\} & 1,048105    \\ \hline
  \{'arah': 'C', 'kecepatan': 131\} & 1,039553    \\ \hline
  \{'arah': 'D', 'kecepatan': 172\} & 1,060164    \\ \hline
  \{'arah': 'C', 'kecepatan': 166\} & 1,062550    \\ \hline
  \{'arah': 'E', 'kecepatan': 47\}  & 1,047506    \\ \hline
  \{'arah': 'E', 'kecepatan': 193\} & 1,052148    \\ \hline
  \{'arah': 'A', 'kecepatan': 54\}  & 1,062337    \\ \hline
  \{'arah': 'B', 'kecepatan': 158\} & 1,066952    \\ \hline
  Average Delay Time                & 1,059223806 \\ \hline
  \end{tabular}
\end{table}

Tabel \ref{tbl:delayBluetoothJSON2} menampilkan pengiriman JSON yang terdiri dari 2 \emph{key}-\emph{value}, yaitu arah dan kecepatan. Dalam pengujian kali ini, data dikirimkan sebanyak 31 kali secara berturut-turut dengan penambahan waktu \emph{delay} sebesar 1.1 detik setiap kali mengirimkan data. Hal ini dilakukan agar ESP32 dapat menerima data dengan baik dan berhasil memisahkan (\emph{deserialize}) serta memasukkan kedua nilai tersebut sesuai dengan variabel yang telah ditentukan. Hasil dari pengujian ini menunjukkan bahwa waktu pengiriman rata-rata dari laptop menuju ESP32 melalui Bluetooth adalah sebesar 1.059223806 detik.

\begin{table}[!h]
\centering
  \caption{Pengujian Waktu Delay Pengiriman Data JSON Berisi 1 Nilai Melalui Bluetooth}
  \label{tbl:delayBluetoothJSON1}
  \begin{tabular}{|c|c|}
  \hline
  Data               & Delay Time  \\ \hline
  \{'arah': 'B'\}    & 1,063216    \\ \hline
  \{'arah': 'D'\}    & 1,050613    \\ \hline
  \{'arah': 'E'\}    & 1,056152    \\ \hline
  \{'arah': 'A'\}    & 1,052444    \\ \hline
  \{'arah': 'C'\}    & 1,039226    \\ \hline
  \{'arah': 'C'\}    & 1,138351    \\ \hline
  \{'arah': 'B'\}    & 1,035915    \\ \hline
  \{'arah': 'D'\}    & 1,056977    \\ \hline
  \{'arah': 'C'\}    & 1,049147    \\ \hline
  \{'arah': 'D'\}    & 1,131767    \\ \hline
  \{'arah': 'C'\}    & 1,137578    \\ \hline
  \{'arah': 'B'\}    & 1,054065    \\ \hline
  \{'arah': 'E'\}    & 1,056702    \\ \hline
  \{'arah': 'E'\}    & 1,052598    \\ \hline
  \{'arah': 'E'\}    & 1,048436    \\ \hline
  \{'arah': 'C'\}    & 1,054793    \\ \hline
  \{'arah': 'A'\}    & 1,059857    \\ \hline
  \{'arah': 'E'\}    & 1,054579    \\ \hline
  \{'arah': 'E'\}    & 1,129237    \\ \hline
  \{'arah': 'E'\}    & 1,031556    \\ \hline
  \{'arah': 'D'\}    & 1,062512    \\ \hline
  \{'arah': 'C'\}    & 1,134129    \\ \hline
  \{'arah': 'D'\}    & 1,11367     \\ \hline
  \{'arah': 'A'\}    & 1,051385    \\ \hline
  \{'arah': 'E'\}    & 1,120536    \\ \hline
  \{'arah': 'B'\}    & 1,116529    \\ \hline
  \{'arah': 'A'\}    & 1,060862    \\ \hline
  \{'arah': 'D'\}    & 1,057455    \\ \hline
  \{'arah': 'C'\}    & 1,060398    \\ \hline
  \{'arah': 'C'\}    & 1,04322     \\ \hline
  \{'arah': 'C'\}    & 1,049035    \\ \hline
  \{'arah': 'C'\}    & 1,101292    \\ \hline
  \{'arah': 'D'\}    & 1,016585    \\ \hline
  \{'arah': 'B'\}    & 1,045899    \\ \hline
  \{'arah': 'D'\}    & 1,044521    \\ \hline
  Average Delay Time & 1,069463914 \\ \hline
  \end{tabular}
\end{table}

Tabel \ref{tbl:delayBluetoothJSON1} menampilkan pengiriman JSON yang terdiri dari 1 \emph{key}-\emph{value}, yaitu arah. Dalam pengujian kali ini, data dikirimkan sebanyak 35 kali secara berturut-turut dengan penambahan waktu \emph{delay} sebesar 1 detik setiap kali mengirimkan data. Hal ini dilakukan agar ESP32 dapat menerima data dengan baik dan berhasil memisahkan (\emph{deserialize}) serta memasukkan nilai tersebut sesuai dengan variabel yang telah ditentukan. Hasil dari pengujian ini menunjukkan waktu pengiriman rata-rata dari laptop menuju ESP32 melalui Bluetooth sebesar 1.069463914 detik.

Dari kedua cara pengujian tersebut, didapatkan bahwa waktu \emph{delay} antara proses pengiriman JSON yang mengandung 2 \emph{key}-\emph{value} dibandingkan dengan JSON yang hanya berisikan 1 \emph{key}-\emph{value} tidak terlalu signifikan. Saat mengirimkan JSON yang berisikan 2 \emph{key}-\emph{value}, pengujian menunjukkan adanya \emph{delay} sebesar 1.059223806 detik. Jika dibandingkan dengan pengujian yang mengirimkan JSON dengan 1 \emph{key}-\emph{value}, waktu \emph{delay} yang tercatat adalah sebesar 1.069463914 detik. Terdapat anomali pada pengujian ini, karena waktu delay pada saat mengirimkan data JSON yang berisikan 2 \emph{key}-\emph{value} lebih kecil jika dibandingkan dengan saat mengirimkan data JSON yang berisikan 1 \emph{key}-\emph{value} walaupun perbedaan waktu \emph{delay}-nya tidak terlalu signifikan.

\subsection{Pengujian Waktu \emph{Delay} Pengiriman Data String Melalui \emph{Access Point} WiFi}

Pengujian waktu \emph{delay} pengiriman data ini dilakukan dengan cara mengirimkan data berupa String dari laptop menuji ESP32 melalui WiFi. ESP32 diatur sebagai \emph{Access Point}. Data yang dikirimkan adalah data arah dan kecepatan yang dipisahkan dengan simbol koma seperti yang dapat dilihat pada Persamaan \ref{eq:string-2Data}

\begin{equation}
  \label{eq:string-2Data}
    Arah(char),Kecepatan(integer)
\end{equation}

Variabel arah memiliki tipe data char yang digunakan untuk menentukan arah gerak dari motor kursi roda serta variabel kecepatan yang memiliki tipe data integer yang akan menentukan kecepatan maksimal dari rotasi motor kursi roda. Hasil dari pengujian waktu \emph{delay} pengiriman data String melalui WiFi dapat dilihat pada Tabel \ref{tbl:delayWiFi2} dan Tabel \ref{tbl:delayWiFi1}

\begin{table}[!h]
\centering
  \caption{Pengujian Waktu Delay Pengiriman Data String Berisi 2 Nilai Melalui Access Point WiFi}
  \label{tbl:delayWiFi2}
  \begin{tabular}{|ccc|c|}
  \hline
  \multicolumn{1}{|c|}{Data}  & \multicolumn{1}{c|}{Timestamp Sent}  & Timestamp Received & Delay Time  \\ \hline
  \multicolumn{1}{|c|}{D,175} & \multicolumn{1}{c|}{22:34:36.127972} & 22:34:37.159       & 1,031028    \\ \hline
  \multicolumn{1}{|c|}{D,220} & \multicolumn{1}{c|}{22:34:37.428370} & 22:34:38.433       & 1,00463     \\ \hline
  \multicolumn{1}{|c|}{E,164} & \multicolumn{1}{c|}{22:34:38.728955} & 22:34:39.760       & 1,031045    \\ \hline
  \multicolumn{1}{|c|}{E,109} & \multicolumn{1}{c|}{22:34:40.029352} & 22:34:41.068       & 1,038648    \\ \hline
  \multicolumn{1}{|c|}{D,187} & \multicolumn{1}{c|}{22:34:41.329739} & 22:34:42.371       & 1,041261    \\ \hline
  \multicolumn{1}{|c|}{A,174} & \multicolumn{1}{c|}{22:34:42.630322} & 22:34:43.649       & 1,018678    \\ \hline
  \multicolumn{1}{|c|}{A,176} & \multicolumn{1}{c|}{22:34:43.931092} & 22:34:44.957       & 1,025908    \\ \hline
  \multicolumn{1}{|c|}{C,245} & \multicolumn{1}{c|}{22:34:45.231675} & 22:34:46.276       & 1,044325    \\ \hline
  \multicolumn{1}{|c|}{E,189} & \multicolumn{1}{c|}{22:34:46.532305} & 22:34:47.821       & 1,288695    \\ \hline
  \multicolumn{1}{|c|}{B,149} & \multicolumn{1}{c|}{22:34:47.832930} & 22:34:48.865       & 1,03207     \\ \hline
  \multicolumn{1}{|c|}{D,113} & \multicolumn{1}{c|}{22:34:49.133646} & 22:34:50.179       & 1,045354    \\ \hline
  \multicolumn{1}{|c|}{A,150} & \multicolumn{1}{c|}{22:34:50.434051} & 22:34:51.482       & 1,047949    \\ \hline
  \multicolumn{1}{|c|}{A,216} & \multicolumn{1}{c|}{22:34:51.734834} & 22:34:52.776       & 1,041166    \\ \hline
  \multicolumn{1}{|c|}{B,21}  & \multicolumn{1}{c|}{22:34:53.035137} & 22:34:54.210       & 1,174863    \\ \hline
  \multicolumn{1}{|c|}{B,153} & \multicolumn{1}{c|}{22:34:54.335986} & 22:34:55.466       & 1,130014    \\ \hline
  \multicolumn{1}{|c|}{E,214} & \multicolumn{1}{c|}{22:34:55.636653} & 22:34:56.679       & 1,042347    \\ \hline
  \multicolumn{1}{|c|}{C,231} & \multicolumn{1}{c|}{22:34:56.937379} & 22:34:57.974       & 1,036621    \\ \hline
  \multicolumn{1}{|c|}{E,41}  & \multicolumn{1}{c|}{22:34:58.237752} & 22:34:59.514       & 1,276248    \\ \hline
  \multicolumn{1}{|c|}{B,210} & \multicolumn{1}{c|}{22:34:59.538330} & 22:35:00.582       & 1,04367     \\ \hline
  \multicolumn{1}{|c|}{C,38}  & \multicolumn{1}{c|}{22:35:00.838618} & 22:35:01.871       & 1,032382    \\ \hline
  \multicolumn{1}{|c|}{E,83}  & \multicolumn{1}{c|}{22:35:02.139088} & 22:35:03.182       & 1,042912    \\ \hline
  \multicolumn{1}{|c|}{A,52}  & \multicolumn{1}{c|}{22:35:03.439819} & 22:35:04.441       & 1,001181    \\ \hline
  \multicolumn{1}{|c|}{A,106} & \multicolumn{1}{c|}{22:35:04.740350} & 22:35:05.764       & 1,02365     \\ \hline
  \multicolumn{1}{|c|}{C,255} & \multicolumn{1}{c|}{22:35:06.040967} & 22:35:07.069       & 1,028033    \\ \hline
  \multicolumn{1}{|c|}{B,205} & \multicolumn{1}{c|}{22:35:07.341426} & 22:35:08.380       & 1,038574    \\ \hline
  \multicolumn{1}{|c|}{C,4}   & \multicolumn{1}{c|}{22:35:08.642326} & 22:35:09.764       & 1,121674    \\ \hline
  \multicolumn{1}{|c|}{C,158} & \multicolumn{1}{c|}{22:35:09.942969} & 22:35:10.994       & 1,051031    \\ \hline
  \multicolumn{1}{|c|}{A,64}  & \multicolumn{1}{c|}{22:35:11.243758} & 22:35:12.268       & 1,024242    \\ \hline
  \multicolumn{1}{|c|}{E,132} & \multicolumn{1}{c|}{22:35:12.544406} & 22:35:13.581       & 1,036594    \\ \hline
  \multicolumn{1}{|c|}{D,248} & \multicolumn{1}{c|}{22:35:13.844881} & 22:35:14.871       & 1,026119    \\ \hline
  \multicolumn{1}{|c|}{B,194} & \multicolumn{1}{c|}{22:35:15.145789} & 22:35:16.175       & 1,029211    \\ \hline
  \multicolumn{3}{|c|}{Average Delay Time}                                                & 1,059681387 \\ \hline
  \end{tabular}
\end{table}

Tabel \ref{tbl:delayWiFi2} menampilkan pengiriman data string yang terdiri dari 2 nilai dan dipisahkan dengan simbol koma (","). Dalam pengujian kali ini, data dikirimkan sebanyak 31 kali secara berturut-turut dengan penambahan waktu \emph{delay} sebesar 1.3 detik setiap kali mengirimkan data. Hal ini dilakukan agar ESP32 dapat menerima data dengan baik dan berhasil memisahkan dan memasukkan kedua nilai tersebut sesuai dengan variabel yang telah ditentukan. Hasil dari pengujian ini menunjukkan bahwa waktu pengiriman rata-rata dari laptop menuju ESP32 melalui WiFi adalah sebesar 1.059681387 detik.

\begin{table}[!h]
  \caption{Pengujian Waktu Delay Pengiriman Data String Berisi 1 Nilai Melalui Access Point WiFi}
  \label{tbl:delayWiFi1}
  \begin{tabular}{|ccc|c|}
  \hline
  \multicolumn{1}{|c|}{Data} & \multicolumn{1}{c|}{Timestamp Sent}  & Timestamp Received & Delay Time    \\ \hline
  \multicolumn{1}{|c|}{C}    & \multicolumn{1}{c|}{21:05:44.335642} & 21:05:44.386       & 0,050358      \\ \hline
  \multicolumn{1}{|c|}{E}    & \multicolumn{1}{c|}{21:05:44.380617} & 21:05:44.433       & 0,052383      \\ \hline
  \multicolumn{1}{|c|}{B}    & \multicolumn{1}{c|}{21:05:44.474109} & 21:05:44.526       & 0,051891      \\ \hline
  \multicolumn{1}{|c|}{E}    & \multicolumn{1}{c|}{21:05:44.544724} & 21:05:44.572       & 0,027276      \\ \hline
  \multicolumn{1}{|c|}{E}    & \multicolumn{1}{c|}{21:05:44.674358} & 21:05:44.682       & 0,007642      \\ \hline
  \multicolumn{1}{|c|}{E}    & \multicolumn{1}{c|}{21:05:44.684230} & 21:05:44.728       & 0,04377       \\ \hline
  \multicolumn{1}{|c|}{E}    & \multicolumn{1}{c|}{21:05:44.787668} & 21:05:44.822       & 0,034332      \\ \hline
  \multicolumn{1}{|c|}{E}    & \multicolumn{1}{c|}{21:05:44.834278} & 21:05:44.854       & 0,019722      \\ \hline
  \multicolumn{1}{|c|}{C}    & \multicolumn{1}{c|}{21:05:44.847274} & 21:05:44.899       & 0,051726      \\ \hline
  \multicolumn{1}{|c|}{A}    & \multicolumn{1}{c|}{21:05:44.855957} & 21:05:44.899       & 0,043043      \\ \hline
  \multicolumn{1}{|c|}{C}    & \multicolumn{1}{c|}{21:05:44.968575} & 21:05:44.977       & 0,008425      \\ \hline
  \multicolumn{1}{|c|}{D}    & \multicolumn{1}{c|}{21:05:44.988174} & 21:05:45.024       & 0,035826      \\ \hline
  \multicolumn{1}{|c|}{C}    & \multicolumn{1}{c|}{21:05:45.018276} & 21:05:45.062       & 0,043724      \\ \hline
  \multicolumn{1}{|c|}{A}    & \multicolumn{1}{c|}{21:05:45.054604} & 21:05:45.100       & 0,045396      \\ \hline
  \multicolumn{1}{|c|}{A}    & \multicolumn{1}{c|}{21:05:45.069057} & 21:05:45.100       & 0,030943      \\ \hline
  \multicolumn{1}{|c|}{C}    & \multicolumn{1}{c|}{21:05:45.123548} & 21:05:45.147       & 0,023452      \\ \hline
  \multicolumn{1}{|c|}{A}    & \multicolumn{1}{c|}{21:05:45.144644} & 21:05:45.194       & 0,049356      \\ \hline
  \multicolumn{1}{|c|}{E}    & \multicolumn{1}{c|}{21:05:45.164004} & 21:05:45.194       & 0,029996      \\ \hline
  \multicolumn{1}{|c|}{A}    & \multicolumn{1}{c|}{21:05:45.218556} & 21:05:45.270       & 0,051444      \\ \hline
  \multicolumn{1}{|c|}{B}    & \multicolumn{1}{c|}{21:05:45.245502} & 21:05:45.270       & 0,024498      \\ \hline
  \multicolumn{1}{|c|}{C}    & \multicolumn{1}{c|}{21:05:45.347689} & 21:05:45.373       & 0,025311      \\ \hline
  \multicolumn{1}{|c|}{C}    & \multicolumn{1}{c|}{21:05:45.355225} & 21:05:45.373       & 0,017775      \\ \hline
  \multicolumn{1}{|c|}{E}    & \multicolumn{1}{c|}{21:05:45.366221} & 21:05:45.410       & 0,043779      \\ \hline
  \multicolumn{1}{|c|}{A}    & \multicolumn{1}{c|}{21:05:45.413087} & 21:05:45.456       & 0,042913      \\ \hline
  \multicolumn{1}{|c|}{C}    & \multicolumn{1}{c|}{21:05:45.458212} & 21:05:45.492       & 0,033788      \\ \hline
  \multicolumn{1}{|c|}{C}    & \multicolumn{1}{c|}{21:05:45.482171} & 21:05:45.534       & 0,051829      \\ \hline
  \multicolumn{1}{|c|}{A}    & \multicolumn{1}{c|}{21:05:45.529777} & 21:05:45.579       & 0,049223      \\ \hline
  \multicolumn{1}{|c|}{C}    & \multicolumn{1}{c|}{21:05:45.603062} & 21:05:45.626       & 0,022938      \\ \hline
  \multicolumn{1}{|c|}{E}    & \multicolumn{1}{c|}{21:05:45.670848} & 21:05:45.710       & 0,039152      \\ \hline
  \multicolumn{1}{|c|}{D}    & \multicolumn{1}{c|}{21:05:45.682481} & 21:05:45.710       & 0,027519      \\ \hline
  \multicolumn{1}{|c|}{B}    & \multicolumn{1}{c|}{21:05:45.703240} & 21:05:45.749       & 0,04576       \\ \hline
  \multicolumn{1}{|c|}{D}    & \multicolumn{1}{c|}{21:05:45.800390} & 21:05:45.843       & 0,04261       \\ \hline
  \multicolumn{1}{|c|}{B}    & \multicolumn{1}{c|}{21:05:45.809048} & 21:05:45.843       & 0,033952      \\ \hline
  \multicolumn{1}{|c|}{E}    & \multicolumn{1}{c|}{21:05:45.858800} & 21:05:45.875       & 0,0162        \\ \hline
  \multicolumn{1}{|c|}{D}    & \multicolumn{1}{c|}{21:05:45.868054} & 21:05:45.875       & 0,006946      \\ \hline
  \multicolumn{3}{|c|}{Average Delay Time}                                               & 0,03499708571 \\ \hline
  \end{tabular}
\end{table}

Tabel \ref{tbl:delayWiFi1} menampilkan pengiriman data string yang terdiri dari 1 nilai, yaitu arah. Dalam pengujian kali ini, data dikirimkan sebanyak 35 kali secara berturut-turut tanpa penambahan waktu \emph{delay} setiap kali mengirimkan data. Hasil dari pengujian ini menunjukkan bahwa waktu pengiriman rata-rata dari laptop menuju ESP32 melalui WiFi adalah sebesar 0.03499708571 detik.

Ditemukan perbedaan waktu \emph{delay} yang cukup signifikan antara proses pengiriman string yang mengandung 2 nilai dibandingkan dengan string yang hanya berisikan 1 nilai. Saat mengirimkan string yang memuat 2 data, pengujian menunjukkan adanya waktu \emph{delay} sekitar 1.059681387 detik. Dalam perbandingan dengan pengujian yang melibatkan string dengan 1 nilai, waktu yang \emph{delay} yang tercatat hanya 0.03499708571 detik. Penting untuk diperhatikan bahwa terdapat kekurangan saat mengirimkan data string yang mengandung 2 nilai, yaitu adanya penambahan \emph{delay} sebesar 1.3 detik setiap kali pengiriman data dilakukan. Hal ini dilakukan agar ESP32 dapat menerima data secara optimal. Oleh karena itu, dapat disimpulkan bahwa dalam konteks pengiriman data melalui WiFi, lebih disarankan untuk mengirimkan string dengan hanya 1 nilai guna menghindari \emph{delay} tambahan yang dapat mempengaruhi efisiensi dan kecepatan transmiri data secara keseluruhan.

\subsection{Pengujian Waktu \emph{Delay} Pengiriman Data JSON Melalui \emph{Access Point} WiFi}

Data JSON terdiri dari 2 bagian, yaitu \emph{key} dan \emph{value}. Terdapat 2 \emph{key}, yaitu arah dan kecepatan. \emph{Key} arah berisi \emph{value} dengan tipe data char yang digunakan untuk menentukan arah gerak dari motor kursi roda dan \emph{key} kecepatan berisi \emph{value} dengan tipe data integer yang digunakan untuk menentukan kecepatan maksimal dari rotasi motor kursi roda. Hasil dari pengujian waktu \emph{delay} pengiriman data JSON dapat dilihat pada Tabel \ref{tbl:delayWiFiJSON} dan Tabel \ref{tbl:delayWiFiJSON1}.

\begin{table}[!h]
\centering
  \caption{Pengujian Waktu Delay Pengiriman Data JSON Berisi 2 Nilai Melalui Access Point WiFi}
  \label{tbl:delayWiFiJSON}
  \begin{tabular}{|c|c|}
  \hline
  Data                              & Delay Time  \\ \hline
  \{'arah': 'C', 'kecepatan': 89\}  & 1,033033    \\ \hline
  \{'arah': 'C', 'kecepatan': 103\} & 1,08969     \\ \hline
  \{'arah': 'C', 'kecepatan': 164\} & 1,047804    \\ \hline
  \{'arah': 'A', 'kecepatan': 101\} & 1,023346    \\ \hline
  \{'arah': 'A', 'kecepatan': 152\} & 1,034008    \\ \hline
  \{'arah': 'A', 'kecepatan': 19\}  & 1,03826     \\ \hline
  \{'arah': 'A', 'kecepatan': 198\} & 1,034545    \\ \hline
  \{'arah': 'A', 'kecepatan': 211\} & 1,039378    \\ \hline
  \{'arah': 'E', 'kecepatan': 238\} & 1,036928    \\ \hline
  \{'arah': 'D', 'kecepatan': 30\}  & 1,168465    \\ \hline
  \{'arah': 'D', 'kecepatan': 112\} & 1,024896    \\ \hline
  \{'arah': 'D', 'kecepatan': 7\}   & 1,003383    \\ \hline
  \{'arah': 'C', 'kecepatan': 66\}  & 1,038815    \\ \hline
  \{'arah': 'A', 'kecepatan': 105\} & 1,042416    \\ \hline
  \{'arah': 'B', 'kecepatan': 117\} & 1,016901    \\ \hline
  \{'arah': 'D', 'kecepatan': 223\} & 1,027115    \\ \hline
  \{'arah': 'A', 'kecepatan': 245\} & 1,004333    \\ \hline
  \{'arah': 'A', 'kecepatan': 15\}  & 1,038492    \\ \hline
  \{'arah': 'D', 'kecepatan': 156\} & 1,030852    \\ \hline
  \{'arah': 'C', 'kecepatan': 118\} & 1,034787    \\ \hline
  \{'arah': 'B', 'kecepatan': 200\} & 1,036985    \\ \hline
  \{'arah': 'E', 'kecepatan': 159\} & 1,01237     \\ \hline
  \{'arah': 'D', 'kecepatan': 39\}  & 1,006606    \\ \hline
  \{'arah': 'C', 'kecepatan': 200\} & 1,040129    \\ \hline
  \{'arah': 'B', 'kecepatan': 179\} & 1,026281    \\ \hline
  \{'arah': 'E', 'kecepatan': 55\}  & 1,044936    \\ \hline
  \{'arah': 'E', 'kecepatan': 228\} & 1,025894    \\ \hline
  \{'arah': 'C', 'kecepatan': 38\}  & 1,040572    \\ \hline
  \{'arah': 'D', 'kecepatan': 38\}  & 1,041131    \\ \hline
  \{'arah': 'C', 'kecepatan': 192\} & 1,045332    \\ \hline
  \{'arah': 'E', 'kecepatan': 56\}  & 1,036805    \\ \hline
  Average Delay Time                & 1,037564129 \\ \hline
  \end{tabular}
\end{table}

Tabel \ref{tbl:delayWiFiJSON} menampilkan pengiriman JSON yang terdiri dari 2 \emph{key}-\emph{value}, yaitu arah dan kecepatan. Dalam pengujian kali ini, data dikirimkan sebanyak 31 kali secara berturut-turut dengan penambahan waktu \emph{delay} sebesar 1.5 detik setiap kali mengirimkan data. Hal ini dilakukan agar ESP32 dapat menerima data dengan baik dan berhasil memisahkan (\emph{deserialize}) serta memasukkan kedua nilai tersebut sesuai dengan variabel yang telah ditentukan. Hasil dari pengujian ini menunjukkan bahwa waktu pengiriman rata-rata dari laptop menuju ESP32 melalui Bluetooth adalah sebesar 1,037564129 detik.

\begin{table}[!h]
\centering
  \caption{Pengujian Waktu Delay Pengiriman Data JSON Berisi 1 Nilai Melalui Access Point WiFi}
  \label{tbl:delayWiFiJSON1}
  \begin{tabular}{|c|c|}
  \hline
  Data               & Delay Time  \\ \hline
  \{'arah': 'C'\}    & 1,031241    \\ \hline
  \{'arah': 'B'\}    & 1,048823    \\ \hline
  \{'arah': 'C'\}    & 1,049194    \\ \hline
  \{'arah': 'A'\}    & 1,011389    \\ \hline
  \{'arah': 'C'\}    & 1,030396    \\ \hline
  \{'arah': 'A'\}    & 1,005565    \\ \hline
  \{'arah': 'C'\}    & 1,040566    \\ \hline
  \{'arah': 'D'\}    & 1,045321    \\ \hline
  \{'arah': 'C'\}    & 1,037305    \\ \hline
  \{'arah': 'E'\}    & 1,035833    \\ \hline
  \{'arah': 'C'\}    & 1,034725    \\ \hline
  \{'arah': 'C'\}    & 1,04032     \\ \hline
  \{'arah': 'D'\}    & 1,041068    \\ \hline
  \{'arah': 'B'\}    & 1,057561    \\ \hline
  \{'arah': 'B'\}    & 1,03344     \\ \hline
  \{'arah': 'A'\}    & 1,026151    \\ \hline
  \{'arah': 'C'\}    & 1,046144    \\ \hline
  \{'arah': 'B'\}    & 1,005782    \\ \hline
  \{'arah': 'A'\}    & 1,051424    \\ \hline
  \{'arah': 'E'\}    & 1,033207    \\ \hline
  \{'arah': 'B'\}    & 1,024771    \\ \hline
  \{'arah': 'E'\}    & 1,036355    \\ \hline
  \{'arah': 'D'\}    & 1,038672    \\ \hline
  \{'arah': 'C'\}    & 1,034598    \\ \hline
  \{'arah': 'B'\}    & 1,050191    \\ \hline
  \{'arah': 'D'\}    & 1,041062    \\ \hline
  \{'arah': 'B'\}    & 1,047551    \\ \hline
  \{'arah': 'D'\}    & 1,050521    \\ \hline
  \{'arah': 'E'\}    & 1,038938    \\ \hline
  \{'arah': 'E'\}    & 1,043145    \\ \hline
  \{'arah': 'E'\}    & 1,045377    \\ \hline
  \{'arah': 'D'\}    & 1,043801    \\ \hline
  \{'arah': 'C'\}    & 1,045352    \\ \hline
  \{'arah': 'B'\}    & 1,040307    \\ \hline
  \{'arah': 'C'\}    & 1,072756    \\ \hline
  Average Delay Time & 1,038824343 \\ \hline
  \end{tabular}
\end{table}

Tabel \ref{tbl:delayWiFiJSON1} menampilkan pengiriman JSON yang terdiri dari 1 \emph{key}-\emph{value}, yaitu arah. Dalam pengujian kali ini, data dikirimkan sebanyak 35 kali secara berturut-turut dengan penambahan waktu \emph{delay} sebesar 1.5 detik setiap kali mengirimkan data. Hal ini dilakukan agar ESP32 dapat menerima data dengan baik dan berhasil memisahkan (\emph{deserialize}) serta memasukkan nilai tersebut sesuai dengan variabel yang telah ditentukan. Hasil dari pengujian ini menunjukkan waktu pengiriman rata-rata dari laptop menuju ESP32 melalui Bluetooth sebesar 1.038824343 detik.

Dari kedua cara pengujian tersebut, didapatkan bahwa waktu \emph{delay} antara proses pengiriman JSON yang mengandung 2 \emph{key}-\emph{value} dibandingkan dengan JSON yang hanya berisikan 1 \emph{key}-\emph{value} tidak terlalu signifikan. Saat mengirimkan JSON yang berisikan 2 \emph{key}-\emph{value}, pengujian menunjukkan adanya \emph{delay} sebesar 1,037564129 detik. Jika dibandingkan dengan pengujian yang mengirimkan JSON dengan 1 \emph{key}-\emph{value}, waktu \emph{delay} yang tercatat adalah sebesar 1.038824343 detik. Terdapat anomali pada pengujian ini, karena waktu delay pada saat mengirimkan data JSON yang berisikan 2 \emph{key}-\emph{value} lebih kecil jika dibandingkan dengan saat mengirimkan data JSON yang berisikan 1 \emph{key}-\emph{value} walaupun perbedaan waktu \emph{delay}-nya tidak terlalu signifikan.
