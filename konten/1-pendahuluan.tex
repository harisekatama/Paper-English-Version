% Ubah judul dan label berikut sesuai dengan yang diinginkan.
\section{Pendahuluan}
\label{sec:pendahuluan}

% Ubah paragraf-paragraf pada bagian ini sesuai dengan yang diinginkan.

Menurut Kamus Besar Bahasa Indonesia, lumpuh merupakan melemahnya fungsi anggota badan sehingga tidak bertenaga atau tidak dapat digerakkan lagi sebagaimana mestinya \cite{Daring_2016}. Otot beserta tulang, saraf, serta jaringan penghubung antara otot, tulang dan saraf memiliki peran yang penting dalam mengendalikan gerak tubuh manusia. Apabila salah satu jaringan mengalami gangguan makan akan terjadi kelumpulan, baik kelumpuhan sementara maupun kelumpuhan permanen.

Terdapat beberapa kondisi yang dapat mengakibatkan kelumpuhan, seperti penyakit stroke yang dapat menyebabkan kelumpuhan pada salah satu sisi wajah, lengan serta tungkai, \emph{Bell's Palsy} yang dapat menyebabkan kelumpuhan pada salah satu sisi wajah tanpa disertai kelumpuhan pada anggota tubuh yang lain, cedera otak yang dapat memicu kelumpuhan pada setiap bagian tubuh sesuai bagian otak yang rusak, polio yang menyebabkan kelumpuhan pada lengan, tungkai, serta otot pernapasan, dan masih banyak kondisi yang menyebabkan kelumpuhan \cite{Pansawira_2022}.

Seseorang yang mengalami kelumpuhan sering kali mengalami permasalahan dalam hal mobilitas sehari-hari. Mereka memerlukan alat tambahan untuk dapat beraktivitas sehari-hari, salah satunya adalah kursi roda. Hingga saat ini sudah terdapat kursi roda elektrik yang dikendalikan dengan menggunakan \emph{joystick} \cite{choi2019motion}. Akan tetapi penggunaan \emph{joystick} belum dapat menjawab permasalahan dari seseorang yang mengalami kelumpuhan. Karena bagi orang yang mengalami kelumpuhan pada bagian lengan akan kekusahan dalam mengendalikan kursi roda elektrik berjenis ini.

Dalam menghadapi permasalahan kelumpuhan, sangat penting untuk mencari solusi yang dapat meningkatkan kemandirian para penderita. Salah satu pendekatan yang menjanjikan adalah memanfaatkan teknologi canggih, seperti visi komputer yang dapat diintegrasikan dengan sistem tertanam. Dengan menggabungkan kedua teknologi ini, diharapkan dapat diciptakan solusi inovatif yang memungkinkan para penderita kelumpuhan untuk tetap dapat bermobilitas secara mandiri.

Visi komputer merupakan bidang keilmuan yang memungkinkan komputer dapat "melihat" \cite{TIAN20201}. Teknologi ini menggunakan kamera untuk mengidentifikasi, melacak, hingga mengukur target untuk pemrosesan citra lebih lanjut. Visi komputer memberikan kemampuan untuk mengenali dan memahami lingkungan sekitar. Sedangkan sistem tertanam dapat diatur secara personal untuk memenuhi kebutuhan spesifik sesuai dengan permasalahan yang dihadapi. 

Integrasi teknologi ini dapat menjadi solusi inovatif terhadap permasalahan yang dihadapi. Dalam rangka mengatasi tantangan ini, penelitian akan difokuskan pada pengembangan kontroler motor yang dapat secara optimal berinteraksi dengan teknologi visi komputer. Pemilihan ESP32 sebagai mikrokontroler utama menjadi langkah strategis, karena kemampuannya dalam mengatur dengan presisi kerja motor. Tidak hanya berfungsi sebagai kontroler motor, ESP32 juga akan berperan sebagai perangkat penerima data dari komputer yang dilengkapi dengan teknologi visi komputer. Melalui integrasi ini, diharapkan bahwa kontroler motor dapat beroperasi secara sinergis dengan informasi yang diterima dari komputer dan menciptakan sebuah sistem yang efisien dan responsif. 
