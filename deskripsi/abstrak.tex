% Mengubah keterangan `Abstract` ke bahasa indonesia.
% Hapus bagian ini untuk mengembalikan ke format awal.
\renewcommand\abstractname{Abstract}

\begin{abstract}

  % Ubah paragraf berikut sesuai dengan abstrak dari penelitian.
  \emph{Paralysis is a condition where a person experiences a weakening of the functions in their body parts, causing them to lack energy or be unable to move their limbs as they should. There are several conditions that can lead to paralysis, ranging from diseases such as stroke to accidents. Individuals experiencing paralysis often face challenges in their daily mobility. They require additional tools to be able to carry out activities, one of which is a wheelchair. Electric wheelchairs controlled by a joystick have been developed to date. However, the use of a joystick may not address the issues faced by someone experiencing paralysis. This is because individuals with paralysis in their arms may struggle to control this type of electric wheelchair. In this research, a controller for an electric wheelchair has been developed that can be operated through computer vision technology, either with hand poses or head gestures. The integration of this technology can be an innovative solution to the problems being faced. Choosing ESP32 as the main microcontroller is a strategic step due to its ability to precisely control the wheelchair's motor. In addition to functioning as a motor controller, ESP32 also serves as a data receiver from the computer equipped with computer vision technology. From the test results, it is concluded that the transmission from computer vision should be analogized to 1 character letter and transmitted using WiFi. This is done because transmitting data containing 1 letter and using WiFi has the best delay time, which is 0.03499708571 seconds. Through this integration, it is expected that the motor controller can operate synergistically with the information received from the computer, creating an efficient and responsive system.}

\end{abstract}

% Mengubah keterangan `Index terms` ke bahasa indonesia.
% Hapus bagian ini untuk mengembalikan ke format awal.
\renewcommand\IEEEkeywordsname{Keywords}

\begin{IEEEkeywords}

  % Ubah kata-kata berikut sesuai dengan kata kunci dari penelitian.
  \emph{Wheelchair}, \emph{Convolutional Neural Network}, \emph{Mediapipe}, \emph{ESP32}, \emph{WiFi}.

\end{IEEEkeywords}
