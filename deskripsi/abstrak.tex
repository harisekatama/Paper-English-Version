% Mengubah keterangan `Abstract` ke bahasa indonesia.
% Hapus bagian ini untuk mengembalikan ke format awal.
\renewcommand\abstractname{Abstract}

\begin{abstract}

  % Ubah paragraf berikut sesuai dengan abstrak dari penelitian.
  \emph{Paralysis is a condition where individuals experience a weakening of the function in their body parts, resulting in them lacking strength or not being able to move their limbs as they should. There are several conditions that can lead to paralysis, ranging from diseases such as stroke to accidents. Individuals with paralysis often face mobility issues in their daily lives. They require additional tools to be able to carry out activities, one of which is a wheelchair. To date, electric wheelchairs controlled using a joystick have been available. However, the use of a joystick has not been able to address the issues faced by individuals with paralysis. This is because those with paralysis in their arms may struggle to control this type of electric wheelchair. In this research, an electric wheelchair controller has been developed that can be operated through computer vision technology, both with hand poses and head gestures. The integration of this technology can provide an innovative solution to the problems being faced. The selection of ESP32 as the main microcontroller is a strategic step, due to its ability to precisely control motor functions. In addition to functioning as a motor controller, ESP32 also serves as a data receiver from a computer equipped with computer vision technology. From the test results, it is concluded that data transmission from computer vision should be sent in JSON format and transmitted using WiFi. This is done because data transmission in JSON format via WiFi has the best delay time, which is 1.032374783 seconds. Through this integration, it is hoped that the motor controller can operate synergistically with the information received from the computer and create an efficient and responsive system.}

\end{abstract}

% Mengubah keterangan `Index terms` ke bahasa indonesia.
% Hapus bagian ini untuk mengembalikan ke format awal.
\renewcommand\IEEEkeywordsname{Keywords}

\begin{IEEEkeywords}

  % Ubah kata-kata berikut sesuai dengan kata kunci dari penelitian.
  \emph{Wheelchair}, \emph{Convolutional Neural Network}, \emph{Mediapipe}, \emph{ESP32}, \emph{WiFi}.

\end{IEEEkeywords}
